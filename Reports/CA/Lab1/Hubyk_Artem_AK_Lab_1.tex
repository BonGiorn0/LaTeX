\documentclass[12pt]{extarticle}
\usepackage{tempora}
\usepackage[T1, T2A]{fontenc}
\usepackage[utf8]{inputenc}
\usepackage[english, ukrainian]{babel}
\usepackage{geometry}
\usepackage{graphicx}
\usepackage{multirow}
\usepackage{multicol}
\usepackage{float}
\graphicspath{{/home/artem/Pictures}}
\geometry
{
    a4paper,
    left=30mm,
    top=15mm,
    right=20mm,
    bottom=15mm,
}

\begin{document}
\begin{titlepage}
    \begin{center}
        \textbf{\normalsize{\MakeUppercase{
            Міністерство Освіти і науки України
            Національний університет "Львівська політехніка"
        }}}

        \begin{flushright}
        \textbf{ІКНІ}\\
        Кафедра \textbf{ПЗ}
        \end{flushright}
        \vspace{15mm}

        \includegraphics[width=0.4\textwidth]{lpnu_logo.png}

        \vspace*{\fill}

        \textbf{\normalsize{\MakeUppercase{Звіт}}}
            
        До лабораторної роботи №1

        \textbf{на тему:} “Синтез та моделювання 
        шифраторів і дешифраторів та 
        мультиплексорів і демультиплексорів у системі Proteus”

        \textbf{з дисципліни:} “Архітектура комп’ютера”
            
        \vspace*{\fill}

        \begin{flushright}

            \textbf{Лектор:}\\
            доцент кафедри ПЗ\\
            Крук О.Г.\\
            \vspace{12pt}

            \textbf{Виконав:}\\
            студент групи ПЗ-24\\
            Губик А. С.\\
            \vspace{12pt}

            \textbf{Прийняв:}\\
            доцент кафедри ПЗ\\
            Задорожний І. М.\\
        \vspace{12pt}
        \end{flushright}

        Львів -- 2022
            
            
    \end{center}
\end{titlepage}

\subsection*{Тема роботи} 
Синтез та моделювання шифраторів і 
дешифраторів та мультиплексорів і демультиплексорів у системі Proteus

\vspace{12pt}

\subsection*{Мета роботи} 
Закріпити практичні навики 
моделювання логічних схем в середовищі системи програм Proteus; поглибити знання про основні типи комбінаційних схем: шифратори, дешифратори, мультиплексори і демультиплексори; опанувати їх синтез; дослідити роботу синтезованих схем в системі програм Proteus

\subsection*{Індивідуальне завдання}
\begin{center}
    \begin{tabular}{| c | c | c | c | c | c | c | c | c | c | c | c |}
    \hline
    \multirow{3}{*}{N} 
    & a0/z0 & 0 & 1 & 0 & 1 & 0 & 1 & 0 & 1 &
    \multirow{3}{*}{$f_0$, KГц} & 
    \multirow{3}{*}{Пріоритет} \\
    \cline{2-10}
    & a1/z1 & 0 & 0 & 1 & 1 & 0 & 0 & 1 & 1 & &\\
    \cline{2-10}
    & a2/z2 & 0 & 0 & 0 & 0 & 1 & 1 & 1 & 1 & &\\
    \hline
    3 &  & 0 & 0 & $d_0$ & $d_1$ & $d_2$ & $d_3$ & $d_4$ & 0 &
    7 &
    $F_7, F_4, F_1, F_2, F_3, F_6, F_5 $ \\
    \hline
    \end{tabular}
\end{center}

\subsection*{Теоретичні відомості}
Логічний елемент (вентиль) – це електронне коло в інтегральному виконанні, 
яке складається переважно з діодів та транзисторів різного типу і реалізує
 одну з елементарних логічних операцій. Логічна операція називається
  елементарною, якщо вона описується булевою функцією одного або двох
   аргументів.
    
Три наступні логічні елементи реалізують основні логічні операції,
 визначені аксіомами алгебри логіки, і відповідно називаються основними 
 або базовими: інвертор (НЕ/NOT), диз’юнктор (АБО/OR), кон’юнктор (І/AND). Вони утворюють функціонально повну систему, оскільки 
  дозволяють реалізувати довільну логічну функцію будь-якої складності.

  Логічні елементи, крім, 
  звичайно, інвертора, можуть мати до восьми входів. 
  Найбільш наочно кожний логічний елемент описується таблицею 
  істинності або відповідності, яка охоплює всі можливі комбінації 
  вхідних сигналів і відповідні значення на виході. 

  Кожен логічний елемент має свою назву і стандартне умовне 
  графічне позначення, на якому вхід/входи завжди розміщені зліва, 
  а вихід/виходи – справа.

На практиці широкого поширення набули й інші логічні елементи, 
які ще називають комбінаційними, оскільки кожен з них реалізує 
декілька основних логічних операцій: виняткове АБО (XOR), елемент 
еквівалентності (XNOR), елемент Пірса (АБО-НЕ/NOR), елемент Шеффера 
(І-НЕ/NAND).

З принципу двоїстості слідує, що в логічних виразах, які задають будь-яку логічну функцію, 
можна обійтись лише двома типами основних операцій, а саме: АБО та НЕ або ж
 І та НЕ. Таким чином, сукупність логічних елементів АБО та НЕ і аналогічно
  сукупність логічних елементів І та НЕ є функціонально повними системами. 
  На практиці доцільніше замість логічних елементів АБО та НЕ 
  використовувати елемент Пірса, що поєднує їх операції. З тих же
 міркувань замість логічних елементів І та НЕ використовують елемент Шеффера. 

\subsection*{Хід роботи}
\paragraph{1.}

За індивідуальним завдання пріоритети: $F_3, F_7, F_2, F_4, F_1, F_6, F_5$ 

Вирази для проміжних змінних пріоритетного шифратора:

$H_7=F_7$

$H_4=\neg F_7\land F_4 $

$H_1=\neg F_7\land F_4\land F_1 $

$H_2=\neg F_7\land F_4\land F_1\land F_2 $

$H_3=\neg F_7\land F_4\land F_1\land F_2\land F_3 $

$H_6=\neg F_7\land \neg F_4\land \neg F_1\land \neg 
            F_2\land \neg F_3\land F_6 $

$H_5=\neg F_7\land \neg F_4\land \neg F_1\land \neg 
            F_2\land \neg F_3\land \neg F_6\land F_5 $

\vspace{12pt}
Вихідні сигнали пріоритетного шифратора:

$X_0=H_1 \land H_3 \land H_5 \land H_7 $

$X_1=H_2 \land H_3 \land H_6 \land H_7  $

$X_2=H_4 \land H_5 \land H_6 \land H_7 $

\vspace{12pt}
Період цифрового сигналу:

T = 1/f = 1/7000 = 142.8 мкc

\vspace{12pt}
Ширина елементарного імпульсу:

t = Т / 8 = 142.8 / 8 = 17.9 мкс

\vspace{12pt}
Схема шифратора і його графік:
\begin{figure}[H]
    \centering
    \includegraphics[width=0.90\textwidth]{coder.jpg}
    \caption{Шифратор}
\end{figure}
\begin{figure}[H]
    \centering
    \includegraphics[width=0.90\textwidth]{coder_graph.jpg}
    \caption{Графік}
\end{figure}

\paragraph{2.}

Рівняння виходів дешифратора:

$V_0 = \neg z_2 \land \neg z_1 \land \neg z_0$

$V_1 = \neg z_2 \land \neg z_1 \land z_0$

$V_2 = \neg z_2 \land z_1 \land \neg z_0$

$V_3 = \neg z_2 \land z_1 \land z_0$

$V_4 = z_2 \land \neg z_1 \land \neg z_0$

$V_5 = z_2 \land \neg z_1 \land z_0$

$V_6 = z_2 \land z_1 \neg \land z_0$

$V_7 = z_2 \land z_1 \land z_0$

\vspace{12pt}
Частоти сигналів:

$Z_0 = 4*f_0 = 4*7000=28000 кГц $

$Z_1 = 2*f_0 = 2*7000=14000 кГц $

$Z_2 = 1*f_0 = 1*7000=7000 кГц $

\begin{figure}[H]
    \centering
    \includegraphics[width=0.90\textwidth]{encoder.jpg}
    \caption{Дешифратор}
\end{figure}
\begin{figure}[H]
    \centering
    \includegraphics[width=0.90\textwidth]{encoder_graph.jpg}
    \caption{Графік}
\end{figure}

\vspace{12pt}
\paragraph{3.}
Вихідний сигнал D у ДДНФ мультиплексора:

$F = (\neg A0 \land A1 \land \neg A2 \land D0) \lor
    (\neg A0 \land A1 \land A2 \land D1) \lor
    (\neg A0 \land A1 \land A2 \land D2) \lor \break \lor
    (A0 \land \neg A1 \land A2 \land D3) \lor
    (A0 \land A1 \land \neg A2 \land D4)$

Ширина елементарного імпульсу мультиплексора:

T/64 = 17.9/8 = 2.28 мкс
\begin{figure}[H]
    \centering
    \includegraphics[width=0.90\textwidth]{multiplexer.jpg}
    \caption{Мультиплексор}
\end{figure}
\begin{figure}[H]
    \centering
    \includegraphics[width=0.90\textwidth]{multiplexer_graph.jpg}
    \caption{Графік}
\end{figure}
\begin{center}
    \begin{tabular}{| c | c | c | c | c | c | c | c | c |}
    \hline
    a0 & a1 & a2 & x0 & x1 & x2 & x3 & x4 & D \\
    \hline
    0 & 1 & 0 & 1 & 0 & 0 & 0 & 0 & x0\\
    0 & 1 & 1 & 0 & 1 & 0 & 0 & 0 & x1\\
    1 & 0 & 0 & 0 & 0 & 1 & 0 & 0 & x2\\
    1 & 0 & 1 & 0 & 0 & 0 & 1 & 0 & x3\\
    1 & 1 & 0 & 0 & 0 & 0 & 0 & 1 & x4\\
    \hline
    \end{tabular}
\end{center}
\paragraph{4.}
Рівняння до кожного з входів демультиплексора:

$Y_0 = \neg A_0 \land A_1 \land \neg A_2 \land D$

$Y_1 = \neg A_0 \land A_1 \land A_2 \land D$

$Y_2 = A_0 \land \neg A_1 \land \neg A_2 \land D$

$Y_3 = A_0 \land \neg A_1 \land A_2  \land D$

$Y_4 = A_0 \land \neg A_1 \land A_2 \land D$

\begin{figure}[H]
    \centering
    \includegraphics[width=0.90\textwidth]{demultiplexer.jpg}
    \caption{Демультиплексор}
\end{figure}
\begin{figure}[H]
    \centering
    \includegraphics[width=0.90\textwidth]{demultiplexer_graph.jpg}
    \caption{Графік}
\end{figure}

\subsection*{Виснок}
Я розібрався як працювати в Proteus, що таке шифратор, мультиплексор і обернені до них схеми.


\end{document}