\documentclass[12pt]{extarticle}
\usepackage{tempora}
\usepackage[T1, T2A]{fontenc}
\usepackage[utf8]{inputenc}
\usepackage[english, ukrainian]{babel}
\usepackage{geometry}
\usepackage{graphicx}
\usepackage{multirow}
\usepackage{multicol}
\usepackage{float}
\usepackage{tikz}
\graphicspath{{/home/artem/Pictures}}
\geometry
{
    a4paper,
    left=30mm,
    top=15mm,
    right=20mm,
    bottom=15mm,
}

\begin{document}
\begin{titlepage}
    \begin{center}
        \textbf{\normalsize{\MakeUppercase{
            Міністерство Освіти і науки України
            Національний університет "Львівська політехніка"
        }}}

        \begin{flushright}
        \textbf{ІКНІ}\\
        Кафедра \textbf{ПЗ}
        \end{flushright}
        \vspace{15mm}

        \includegraphics[width=0.4\textwidth]{lpnu_logo.png}

        \vspace*{\fill}

        \textbf{\normalsize{\MakeUppercase{Звіт}}}
            
        До розрахункової роботи №2

        \textbf{на тему:} “Подання чисел в комп’ютері”

        \textbf{з дисципліни:} “Архітектура комп’ютера”
            
        \vspace*{\fill}

        \begin{flushright}

            %\textbf{Лектор:}\\
            %доцент кафедри ПЗ\\
            %Крук О.Г.\\
            %\vspace{12pt}

            \textbf{Виконав:}\\
            студент групи ПЗ-24\\
            Губик А. С.\\
            \vspace{12pt}

            %\textbf{Прийняв:}\\
            %доцент кафедри ПЗ\\
            %Задорожний І. М.\\
        \vspace{12pt}
        \end{flushright}

        Львів -- 2023
            
            
    \end{center}
\end{titlepage}

\vspace{12pt}

\subsection*{Індивідуальне завдання}

\begin{center}
    \begin{tabular}{| c | c | c | c |}
    \hline
    Варіант & a & b & c\\
    \hline
    3 & 171 & -101 & -161.49707031250\\
    \hline
    \end{tabular}
\end{center}

\subsection*{Хід роботи}
\paragraph{1.}
Переведемо число \textit{a} в двійкову систему:

$171 = 128 + 32 + 8 + 2 + 1$

$171 = 2^7 + 2^5 + 2^3 + 2^1 + 2^0$
\vspace*{12pt}

Тоді ми маємо встановити одиницею біти 7, 5, 3, 1 і 0:

$171_{10} = 0000000010101011_2 = 00AB_{16}$

\paragraph{2.}
Переведемо число \textit{b} 
в двійкову систему, для початку зробим це з його значенням по модулю:

$101 = 64 + 32 + 4 + 1$

$101 = 2^6 + 2^5 + 2^2 + 2^0$
\vspace*{12pt}

Тоді ми маємо встановити одиницею біти 6, 5, 2 і 0:

 $101_{10} = 0000000001100101_2$

\vspace*{12pt}
Інвертуємо біти і додаємо одиницю:

$-101_{10} = 1111111110011011_2 = FF9B_{16}$


\paragraph{3.}
Додамо a i b:
\begin{center}
    +
    \begin{tabular}{ c }
    0000000010101011\\
    1111111110011011\\
    \hline
    0000000001000110\\
    \end{tabular}
\end{center}

$0000000001000110_2 = 2^6 + 2^2 + 2^1$

$0000000001000110_2 = 70_{10}$


\paragraph{4.}
Переведемо число \textit{с} в двійкову сисетму, 
для початку знайдемо двійкове представлення модуля цілої частини:

$161 = 2^7 + 2^5 + 2^0$

$161_{10} = 0000000010100001_2$

\vspace{12pt}
Потім дробової:
\begin{center}
    \begin{tabular}{ c | l c}
    0 & 74707031250 & x 2\\
    1 & 494140625 & x 2\\
    0 & 98828125 & x 2\\
    1 & 9765625 & x 2\\
    1 & 953125 & x 2\\
    1 & 90625 & x 2\\
    1 & 8125 & x 2\\
    0 & 625 & x 2\\
    1 & 25 & x 2\\
    0 & 5 & x 2\\
    1 & 0
    \end{tabular}
\end{center}

$0,74707031250_{10} = 0,1011111101_2$
\vspace{12pt}

Обчислимо мантису:

$E = 127 + 7 = 134 = 10000110$

\vspace{12pt}
Додамо нулів в кінець щоб було 32 біти і тоді кінцевий результат буде:

                        %1 10000110 01000011011111101000000
$-161,74707031250_{10} = 1 10000110 01000011011111101000000_2 = C321BF40_{16}$

\subsection*{Висновок}

Я навчився переводити цілі числа і числа з рухомою комою з десяткової
системи числення в двійкову та шістнадцяткову, та навпаки.
\end{document}