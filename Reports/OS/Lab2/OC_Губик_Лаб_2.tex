\documentclass[12pt]{extarticle}
\usepackage{tempora}
\usepackage[T1, T2A]{fontenc}
\usepackage[utf8]{inputenc}
\usepackage[english, ukrainian]{babel}
\usepackage{geometry}
\usepackage{graphicx}
\usepackage{multirow}
\usepackage{multicol}
\usepackage{float}
\graphicspath{{/home/artem/Pictures}}
\geometry
{
    a4paper,
    left=30mm,
    top=15mm,
    right=20mm,
    bottom=15mm,
}

\begin{document}
\begin{titlepage}
    \begin{center}
        \textbf{\normalsize{\MakeUppercase{
            Міністерство Освіти і науки України
            Національний університет "Львівська політехніка"
        }}}

        \begin{flushright}
        \textbf{ІКНІ}\\
        Кафедра \textbf{ПЗ}
        \end{flushright}
        \vspace{15mm}

        \includegraphics[width=0.4\textwidth]{lpnu_logo.png}

        \vspace*{\fill}

        \textbf{\normalsize{\MakeUppercase{Звіт}}}
            
        До лабораторної роботи №2

        \textbf{на тему:} “Ознайомлення та керування процесами
         в операційних системах для персонального комп'ютера. Linux та macOS”

        \textbf{з дисципліни:} “Операційні системи”
            
        \vspace*{\fill}

        \begin{flushright}

            \textbf{Лектор:}\\
            старший викладач кафедри ПЗ\\
            Грицай О.Д.\\
            \vspace{12pt}

            \textbf{Виконав:}\\
            студент групи ПЗ-24\\
            Губик А. С.\\
            \vspace{12pt}

            \textbf{Прийняв:}\\
            доцент кафедри ПЗ\\
            Горечко О. М.\\
        \vspace{12pt}
        \end{flushright}

        Львів -- 2023
            
            
    \end{center}
\end{titlepage}

\textbf{Тема роботи:} Ознайомлення та керування процесами в операційних системах для
персонального комп’ютера. Linux та macOS.
\vspace{12pt}

\textbf{Мета роботи:} Ознайомитися з процесами та потоками в операційних системах
Linux, MacOS. Навчитися працювати із системними утилітами, що дають
можливість отримувати інформацію про процеси, потоки, використовувану
ними пам'ять, та іншу необхідну інформацію.

\subsection*{Теоретичні відомості}
Операційна система - це сукупність програм, які призначені для
керування ресурсами комп'ютера й обчислювальними процесами, а також для
організації взаємодії користувача з апаратурою. З іншої сторони, Операційна
система - це програма, яка постійно працює на комп'ютері і, зазвичай,
називається ядром.
Функції операційних систем можна взагальному описати, як:
Керування та розподіл ресурсів
Керування обчислювальними процесами.
Забезпечення взаємодії користувача з апаратурою
Класифікація операційних систем здійснюється відносно різних
характеристик:
а будовою ядра: монолітні, мікроядерні, наноядерні;
за кількістю розрядів даних, що обробляються одночасно: 8-, 16-,
32-, 64-розрядні;
за кількістю програм, що виконуються одночасно: однозадачні,
багатозадачні;
за цільовим пристроєм: для мейнфреймів, для ПК, для мобільних
пристроїв;
за типом інтерфейсу: з текстовим інтерфейсом, з графічним
інтерфейсом;
за кількістю користувачів: однокористувацькі, багатокористувацькі;
за типом використання ресурсів: локальні, мережеві;
за призначенням: для пакетної обробки, інтерактивні, підтримка
реального часу;
за типом ліцензії: комерційна, вільна;
за сімейством: Microsoft Windows, Unix-подібні ОС, Mac OS X
та інші.

\subsection*{Хід роботи}


\paragraph{1.}
Встановити операційні системи Linux та MacOS

\vspace{12pt}
Я встановив Linux поряд з Windows(dual boot).
\paragraph{2.}
За допомого консольних засобів ОС Linux отримати повну інформацію
про процеси.

\begin{figure}[H]
    \centering
    \includegraphics[width=0.80\textwidth]{htop.png}
    \caption{htop}
\end{figure}

\paragraph{3.}
а допомогою утиліт top, htop, qps, System Monitor отримати повну
інформацію про процеси в ОС Linux та MacOS.

\begin{figure}[H]
    \centering
    \includegraphics[width=0.90\textwidth]{top.png}
    \caption{top}
\end{figure}
\begin{figure}[H]
    \centering
    \includegraphics[width=0.90\textwidth]{qps.png}
    \caption{qps}
\end{figure}
\begin{figure}[H]
    \centering
    \includegraphics[width=0.90\textwidth]{gnome_sys_mon.png}
    \caption{GNOME System Monitor}
\end{figure}

\paragraph{4.}
Використовуючи консольні засоби ОС Linux та утиліти змінити
пріоритет виконання процесу.
\begin{figure}[H]
    \centering
    \includegraphics[width=0.90\textwidth]{prior.png}
    \caption{Виставляємо пріорітет в System Monitor}
\end{figure}

\break
\paragraph{5.}
Використовуючи консольні засоби ОС Linux та сторонні утиліти
змінити стан виконання процесу, завершити виконання заданого процесу.

\begin{figure}[H]
    \centering
    \includegraphics[width=0.90\textwidth]{end_task.png}
    \caption{Відновлюємо роботу Edge}
\end{figure}

\vspace{12pt}

\paragraph{6.}
Скомпілювати файл main.cpp представлений у лабораторній роботі No 1
(на MacOS і Linux можна командою: g++ main.cpp -pthread) і запустити
виконуваний файл на різній кількості активних процесорів (ядер). Знайти для
даної програми величини , , при різних вхідних значеннях величини .A S p n
Порівняти результати для різних операційних систем.

\begin{figure}[H]
    \centering
    \includegraphics[width=0.90\textwidth]{core1.png}
    \caption{}
\end{figure}
В сеедньому виходить 925 ms на одному ядрі, назвемо це число $T_1$

\begin{figure}[H]
    \centering
    \includegraphics[width=0.90\textwidth]{core6.png}
    \caption{}
\end{figure}
В сеедньому виходить 282 ms на шести ядраx, назвемо це число $T_6$.

Визначимо реальне прискорення $A$ для цього випадку, за формулою 
\vspace{12pt}

$A = \frac{T_1}{T_6}$

Результатом буде 3.28


\begin{figure}[H]
    \centering
    \includegraphics[width=0.90\textwidth]{core12.png}
    \caption{}
\end{figure}
В середньому виходить 214 ms на дванадцяти ядраx, назвемо це число $T_{12}$

Визначимо реальне прискорення $A$ для цього випадку, за формулою 
\vspace{12pt}

$A = \frac{T_1}{T_{12}}$

Результатом буде 4.32
\vspace{12pt}

Далі ми визначимо $p$:

$A = \frac{1}{p + \frac{1-p}{n}}$

\vspace{12pt}

$\frac{1}{A} = p + \frac{1-p}{n}$

\vspace{12pt}

$\frac{n}{A} = np + 1 - p$

\vspace{12pt}

$\frac{n}{A} = p(n-1) + 1$

\vspace{12pt}

$p = \frac{n}{A(n-1)} - 1$

\vspace{12pt}

Для шести ядер: $p = 0.63$

Для дванадцяти ядер: $p = 0.75$

\vspace{12pt}

Тоді $S$ для шести ядер: $S = 1.44$

Тоді $S$ для дванадцяти ядер: $S = 1.29$

\paragraph{7.} Результати лабораторної роботи оформити у звіт, у висновку надати
порівняння моніторингу процесів у різних системах різними утилітами,
відповідно до індивідуального варіанту.

Варіант 3: копіювання файлів за допомогою cp

\vspace{12pt}

\vspace{12pt}
\begin{figure}[H]
    \centering
    \includegraphics[width=0.90\textwidth]{cp_htop.png}
    \caption{}
\end{figure}
\begin{figure}[H]
    \centering
    \includegraphics[width=0.90\textwidth]{cp_gnome.png}
    \caption{}
\end{figure}
\textbf{Висновок:} Я навчився змінювати параметри процесів та керувати ними в ОС Linux. Щодо 
завдання 6, можемо бачити що лінукс працює швидше на меншій кількості ядер, він 
працює на 6 ядрах так як віндовс на 12, але при збільшенні ядер до 12 прискорення майже не відбувається.
Можна сказати що лінукс розпаралелює програми більш ефективно.
\end{document}
