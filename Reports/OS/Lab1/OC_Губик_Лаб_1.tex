\documentclass[12pt]{extarticle}
\usepackage{tempora}
\usepackage[T1, T2A]{fontenc}
\usepackage[utf8]{inputenc}
\usepackage[english, ukrainian]{babel}
\usepackage{geometry}
\usepackage{graphicx}
\usepackage{multirow}
\usepackage{multicol}
\usepackage{float}
\graphicspath{{/home/artem/Pictures}}
\geometry
{
    a4paper,
    left=30mm,
    top=15mm,
    right=20mm,
    bottom=15mm,
}

\begin{document}
\begin{titlepage}
    \begin{center}
        \textbf{\normalsize{\MakeUppercase{
            Міністерство Освіти і науки України
            Національний університет "Львівська політехніка"
        }}}

        \begin{flushright}
        \textbf{ІКНІ}\\
        Кафедра \textbf{ПЗ}
        \end{flushright}
        \vspace{15mm}

        \includegraphics[width=0.4\textwidth]{lpnu_logo.png}

        \vspace*{\fill}

        \textbf{\normalsize{\MakeUppercase{Звіт}}}
            
        До лабораторної роботи №1

        \textbf{на тему:} “Ознайомлення та керування процесами
         в операційних системах для персонального комп'ютера.”

        \textbf{з дисципліни:} “Операційні системи”
            
        \vspace*{\fill}

        \begin{flushright}

            \textbf{Лектор:}\\
            старший викладач кафедри ПЗ\\
            Грицай О.В.\\
            \vspace{12pt}

            \textbf{Виконав:}\\
            студент групи ПЗ-24\\
            Губик А. С.\\
            \vspace{12pt}

            \textbf{Прийняв:}\\
            доцент кафедри ПЗ\\
            Горечко О. М.\\
        \vspace{12pt}
        \end{flushright}

        Львів -- 2023
            
            
    \end{center}
\end{titlepage}

\textbf{Тема роботи:} Ознайомлення та керування процесами в операційних системах для
персонального комп’ютера. Windows.
\vspace{12pt}

\textbf{Мета роботи:} Ознайомитися з процесами та потоками в операційній системі
Windows. Навчитися працювати із системними утилітами, що дають
можливість отримувати інформацію про процеси, потоки, використовувану
ними пам'ять, та іншу необхідну інформацію.

\subsection*{Теоретичні відомості}
Термін “Операційна система” охоплює багато визначень та функцій.
Частково, це через велику різноманітність комп'ютерних систем, які
нероздільно пов'язані з операційними системами. Основною метою
комп'ютерних систем є виконання програм та полегшення вирішення проблем
користувачів. Комп'ютерна техніка побудована для досягнення цієї мети.
Оскільки саме обладнання не є простим у використанні, розробляються
прикладні програми. Ці програми вимагають певних типових операцій, таких
як керування пристроями вводу-виводу. Потім загальні функції контролю та
розподілу ресурсів об’єднуються в одне програмне забезпечення: операційну
систему. Отже, першим визначенням операційної системи можна вважати:
Операційна система - це сукупність програм, які призначені для
керування ресурсами комп'ютера й обчислювальними процесами, а також для
організації взаємодії користувача з апаратурою.
З іншої сторони, немає загального визначення, що є складовою частиною
операційної системи. Операційна система може містити ті, чи інші функції,
залежно від призначення і виду різних комп'ютерних систем. Тому,
найпоширенішим визначенням є:
Операційна система - це програма, яка постійно працює на комп'ютері і,
зазвичай, називається ядром.
Поряд з ядром існують ще два типи програм: системні програми, які
пов'язані з операційною системою, але не обов'язково є частиною ядра, та
прикладні програми, що включають усі програми, не пов'язані з роботою
системи.

\subsection*{Хід роботи}


\paragraph{1.}
За допомогою утиліти «Диспетчер задач» та Process Explorer отримати
повну інформацію про процеси
\begin{figure}[H]
    \centering
    \includegraphics[width=0.90\textwidth]{task_manager.jpg}
    \caption{Task Manager Screenshot}
\end{figure}

\paragraph{2.}
За допомогою утиліти Process Explorer отримати додаткову інформацію
про процеси та їхні потоки.

\begin{figure}[H]
    \centering
    \includegraphics[width=0.90\textwidth]{process_exp.jpg}
    \caption{Process Explorer Screenshot}
\end{figure}

\paragraph{3.}
Задати
відповідність виконання процесів на окремих ядрах центрального процесора;
виконати завершення процесу

\begin{figure}[H]
    \centering
    \includegraphics[width=0.90\textwidth]{hollow1.jpg}
    \caption{Звичайний стан процесора}
\end{figure}
\begin{figure}[H]
    \centering
    \includegraphics[width=0.90\textwidth]{hollow2.jpg}
    \caption{Стaн процесора після запуску Hollow Knight}
\end{figure}
\begin{figure}[H]
    \centering
    \includegraphics[width=0.90\textwidth]{hollow3.jpg}
    \caption{Стaн процесора після того, 
        як Hollow Knight виділили ядра від CPU5 по CPU9}
\end{figure}
\begin{figure}[H]
    \centering
    \includegraphics[width=0.60\textwidth]{kill.jpg}
    \caption{Завершуємо процес}
\end{figure}

\begin{figure}[H]
    \centering
    \includegraphics[width=0.60\textwidth]{prior.jpg}
    \caption{Виствляємо процес}
\end{figure}

\paragraph{4.}
Використовуючи Process Explorer призупинити процес і відновити його
роботу.
\begin{figure}[H]
    \centering
    \includegraphics[width=0.60\textwidth]{suspend.jpg}
    \caption{Призупиняємо Edge}
\end{figure}
\begin{figure}[H]
    \centering
    \includegraphics[width=0.60\textwidth]{resume.jpg}
    \caption{Відновлюємо роботу Edge}
\end{figure}

\paragraph{5.}
Скомпілювати файл main.cpp представлений нижче і запустити
виконуваний файл на різній кількості активних процесорів (ядер). Знайти для
даної програми величини $A, S, p$ при різних вхідних значеннях величини $n$.

\vspace{12pt}

Зробимо по 3 заміри на різній кількості ядер: на одному ядрі, шести і дванадцяти

\begin{figure}[H]
    \centering
    \includegraphics[width=0.90\textwidth]{core0_1.jpg}
    \caption{}
\end{figure}
\begin{figure}[H]
    \centering
    \includegraphics[width=0.90\textwidth]{core0_2.jpg}
    \caption{}
\end{figure}
\begin{figure}[H]
    \centering
    \includegraphics[width=0.90\textwidth]{core0_3.jpg}
    \caption{}
\end{figure}

В сеедньому виходить 2123 ms на одному ядрі, назвемо це число $T_1$

\begin{figure}[H]
    \centering
    \includegraphics[width=0.90\textwidth]{core6_1.jpg}
    \caption{}
\end{figure}
\begin{figure}[H]
    \centering
    \includegraphics[width=0.90\textwidth]{core6_2.jpg}
    \caption{}
\end{figure}
\begin{figure}[H]
    \centering
    \includegraphics[width=0.90\textwidth]{core6_3.jpg}
    \caption{}
\end{figure}

В сеедньому виходить 1111 ms на шести ядраx, назвемо це число $T_6$.

Визначимо реальне прискорення $A$ для цього випадку, за формулою 
\vspace{12pt}

$A = \frac{T_1}{T_6}$

Результатом буде 1.910

\begin{figure}[H]
    \centering
    \includegraphics[width=0.90\textwidth]{core12_1.jpg}
    \caption{}
\end{figure}
\begin{figure}[H]
    \centering
    \includegraphics[width=0.90\textwidth]{core12_2.jpg}
    \caption{}
\end{figure}
\begin{figure}[H]
    \centering
    \includegraphics[width=0.90\textwidth]{core12_3.jpg}
    \caption{}
\end{figure}

В сеедньому виходить 212 ms на дванадцяти ядраx, назвемо це число $T_{12}$

Визначимо реальне прискорення $A$ для цього випадку, за формулою 
\vspace{12pt}

$A = \frac{T_1}{T_{12}}$

Результатом буде 10.014
\vspace{12pt}

Далі ми визначимо $p$:

$A = \frac{1}{p + \frac{1-p}{n}}$

\vspace{12pt}

$\frac{1}{A} = p + \frac{1-p}{n}$

\vspace{12pt}

$\frac{n}{A} = np + 1 - p$

\vspace{12pt}

$\frac{n}{A} = p(n-1) + 1$

\vspace{12pt}

$p = \frac{n}{A(n-1)} - 1$

\vspace{12pt}

Для шести ядер: $p = 0.37$

Для дванадцяти ядер: $p = 0.9$

\vspace{12pt}

Тоді $S$ для шести ядер: $S = 2.1$

Тоді $S$ для дванадцяти ядер: $S = 1.1$

\paragraph{6.}
Дослідити вплив зміни відповідності ядру на швидкодію процесу.

Варіант 2: стискання файлу

\vspace{12pt}

\vspace{12pt}
\begin{figure}[H]
    \centering
    \includegraphics[width=0.90\textwidth]{zip12.jpg}
    \caption{7zip що працює на 12 ядрах}
\end{figure}
\begin{figure}[H]
    \centering
    \includegraphics[width=0.90\textwidth]{zip1.jpg}
    \caption{7zip що працює на 1 ядрі}
\end{figure}
\textbf{Висновок:} Я навчився змінювати параметри процесів та керувати ними в ОС Windows.
\end{document}
